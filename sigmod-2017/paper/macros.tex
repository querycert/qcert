% macro for linking to the info docs

\newcommand{\fleur}{\ding{95}}
\newcommand{\coqtop}{\text{\href{\coqHTMLBase/index.html}{\fleur}}}
\newcommand{\coqdef}[2]{\text{\href{\coqHTMLBase/\coqBaseModule#1.html\##2}{\fleur}}}


% macros for typesetting rules (expressions of the rule calculus)
\newcommand{\kkeyword}[1]{\textup{\textbf{{#1}}}}

\newcommand{\rit}{\kkeyword{it}}
\newcommand{\renv}{\kkeyword{env}}

\newcommand{\rassert}[1]{\kkeyword{assert}~{#1}}
\newcommand{\rleti}[2]{\kkeyword{let it} = {#1}~\kkeyword{in}~{#2}}
\newcommand{\rlete}[2]{\kkeyword{let env}\;+\!\!= {#1}~\kkeyword{in}~{#2}}
\newcommand{\rmap}[1]{\kkeyword{map}~{#1}}
\newcommand{\rapp}{\:@\:}
\newcommand{\rcast}[1]{\kkeyword{cast~{#1}}}

\newcommand{\qapp}{\:@\:}

\newcommand{\rerr}{\kkeyword{err}}
\newcommand{\ror}{|\!|}

% NRA queries
\newcommand{\NRAEnv}{\texorpdfstring{NRA$^{\!\mbox{\it e}}$}{NRAe}\xspace}
\newcommand{\NRALambda}{\texorpdfstring{NRA$^{\!\mbox{\it $\lambda$}}$}{NRAλ}\xspace}

\newcommand{\qor}{|\!|}
%\newcommand{\qdjoinop}{\raisebox{-2pt}{\ensuremath{\;\overset{d}{\bowtie}}}}
\newcommand{\qdjoinop}{\raisebox{-2pt}{\ensuremath{{\bowtie^d}}}}
\newcommand{\qdjoin}[2]{\ensuremath{\qdjoinop_{\left\langle#1\right\rangle}\!\left(#2\right)}}
%\newcommand{\qmap}[2]{\pi_{\left\langle{#1}\right\rangle}\!\left({#2}\right)}
\newcommand{\qmap}[2]{\chi_{\left\langle{#1}\right\rangle}\!\left({#2}\right)}
\newcommand{\qmapenv}[1]{\chi^e_{\left\langle{#1}\right\rangle}}
\newcommand{\qunnest}[3]{\rho_{#2/\{#1\}}\!\left({#3}\right)}
\newcommand{\qselect}[2]{\sigma\!_{\left\langle{#1}\right\rangle}\!\left({#2}\right)}
\newcommand{\qID}{\kkeyword{In}}
\newcommand{\qENV}{\kkeyword{Env}}
\newcommand{\qcomp}[2]{\ensuremath{\qdjoinop_{\left\langle#1\right\rangle}\!\left(#2\right)}}

% derived rule constructs
\newcommand{\rmapall}[1]{\kkeyword{mapall}~{#1}}
\newcommand{\rmapsnone}[1]{\kkeyword{mapsnone}~{#1}}
\newcommand{\raggregate}{\kkeyword{aggregate}}
\newcommand{\rWW}[1]{\kkeyword{WW}\left({#1}\right)}
\newcommand{\rRAB}[1]{\kkeyword{RAB}\left({#1}\right)}
\newcommand{\rWORLD}{\textit{WORLD}}

% rules to CAMP translation
\newcommand{\rerp}[1]{\llbracket{#1}\rrbracket}

% rule parts
\newcommand{\makerp}[1]{\kkeyword{\emph{#1}}}
\newcommand{\rpwhen}{\makerp{when}}
\newcommand{\rpglobal}{\makerp{global}}
\newcommand{\rpnot}{\makerp{not}}
\newcommand{\rpreturn}{\makerp{return}}

% NNRC expressions
\newcommand{\elet}[3]{\kkeyword{let}~{#1} = {#2}~\kkeyword{in}~{#3}}
\newcommand{\efor}[3]{\{{#3} \mid {#1} \in {#2}\}}
\newcommand{\eif}[3]{{#1}~?~{#2} : {#3}}

% operators
\newcommand{\opunop}{\boxplus}
\newcommand{\opbinop}{\boxtimes}
\newcommand{\opand}{\&\&}
\newcommand{\opnot}{\lnot}
\newcommand{\opflatten}{\textit{flatten}}
\newcommand{\opmergeconcat}{\otimes}
\newcommand{\opconcat}{\oplus}

% NNRC to CAMP translation 
\newcommand{\ntop}[1]{\llbracket{#1}\rrbracket}

% CAMP to NRA translation
\newcommand{\pton}[1]{\llbracket{#1}\rrbracket_r}
\newcommand{\etoa}[1]{\llbracket{#1}\rrbracket_a}
\newcommand{\quntwoop}[2]{\rho_{{#2}/{\{#1\}}}}
\newcommand{\quntwo}[3]{\rho_{{#2}/{\{#1\}}}\left({#3}\right)}

% used as part of the efficient translation from NNRC to CAMP
\newcommand{\rfresh}{\kkeyword{fresh}}

% Lambda queries
\newcommand{\llambda}[2]{\lambda#1.(#2)}
\newcommand{\lmapop}{\kkeyword{map}}
\newcommand{\lselop}{\kkeyword{filter}}
\newcommand{\ldjoinop}{\kkeyword{d-join}}

\newcommand{\lmap}[2]{\lmapop\left(#1\right)#2}
\newcommand{\ldjoin}[2]{\ldjoinop\left(#1\right)#2}
\newcommand{\lsel}[2]{\lselop\left(#1\right)#2}

% used for nralambda to nraenv
\newcommand{\ftoe}[1]{\llbracket{#1}\rrbracket_f}
\newcommand{\ltoe}[1]{\llbracket{#1}\rrbracket_l}

% NRA to NNRC
\newcommand{\nton}[2]{\ensuremath{\llbracket{#2}\rrbracket_{#1}}}
\newcommand{\nfresh}{\kkeyword{fresh}}

% type system
\newcommand{\tnil}{\texttt{NIL}}
\newcommand{\tint}{\texttt{INT}}
\newcommand{\tbool}{\texttt{BOOL}}
\newcommand{\tstring}{\texttt{STRING}}

% operator has type 
\newcommand{\oht}{:}

% data has type 
\newcommand{\dht}{:}

% CAMP pattern has type 
\newcommand{\rht}{:}

% NRA query has type 
\newcommand{\qht}{:}

% NNRC expression has type 
\newcommand{\eht}{:}

% data nil
\newcommand{\dnil}{\textsf{nil}}

\newcommand{\emptyrecord}{\ensuremath{[\,]}}
\newcommand{\true}{\textsf{\small \mbox{true}}\xspace}
\newcommand{\false}{\textsf{\small \mbox{false}}\xspace}
\newcommand{\compatible}{\textsf{\small \mbox{compatible}}\xspace}

\lstdefinelanguage{jrules}{
  morekeywords={aggregate,do,emit,false,groupby,insert,new,rule,then,true,update,when},
  numbers=none, columns=flexible, captionpos=b,
  basicstyle={\small\selectfont\ttfamily}, keywordstyle=\bfseries,
  morecomment=[l]{//},
  morecomment=[s]{/*}{*/},
  xleftmargin=0.5cm,
  abovecaptionskip=0em,
  belowcaptionskip=0em
% ,frame=tlbr,framesep=4pt,framerule=0pt
}

\newcommand{\myparagraph}[1] { \noindent {\textbf {#1.}} }

\newenvironment{gram}{
  \[
  \begin{array}{lcl}
}{
  \end{array}
  \]
}

\newcommand{\Downarrowr}{\Downarrow_r} %rules
\newcommand{\Downarrowa}{\Downarrow_a} %nra+env
\newcommand{\Downarrown}{\Downarrow_n} %nra
\newcommand{\Downarrowc}{\Downarrow_c} %calculus

% Brands

\newcommand{\brand}[1]{\mbox{\sf #1}}
\newcommand{\branded}[2]{{\mbox{\sf brand #1}~({#2})}}
\newcommand{\leftbranded}[1]{{\mbox{\sf brand #1}~}}
\newcommand{\upcast}[2]{(\uparrow \mbox{\sf #1})~{#2}}
\newcommand{\downcast}[2]{(\mbox{\sf #1})~{#2}}
\newcommand{\cast}[2]{(\mbox{\sf #1})~{#2}}
\newcommand{\BrandType}[1]{Brand~\mbox{\sf #1}}
\newcommand{\derivesfrom}[2]{\delta(\mbox{\sf #1},\mbox{\sf #2})}

% Typed Data

\newcommand{\hastype}[3]{{#1} \vdash {#2} \dht {#3}}

% Subtyping

\newcommand{\subtype}[2]{{#1} \preccurlyeq {#2}}
\newcommand{\subtyped}[2]{{#1} \preccurlyeq_{\delta} {#2}}

% ignores stuff

\newcommand{\nra}[1]{\mbox{NRA}(#1)}
\newcommand{\igni}[1]{\mbox{$\cal I$}^i(#1)}
\newcommand{\igne}[1]{\mbox{$\cal I$}^e(#1)}

% environment stuff

%\newcommand{\comment}[1]{\color{red}{#1}}
\newtheorem{definition}{Definition}
\newtheorem{example}{Example}
\newtheorem{theorem}{Theorem}
\newtheorem{corollary}{Corollary}

%% Formattin/font stuff
\renewcommand{\ttdefault}{lmtt}
\vbadness=2000
\newcommand\blueuline[1]{{\underline{\color{blue}{\color{black}{#1}}}}}
