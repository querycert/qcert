\section{Related Work}
\label{sec:related}

There has been renewed interest in the formal verification of
database systems or query
languages~\cite{BenzakenCD14,CheneyU11,malecha2015,MalechaMSW10,ShinnarSH15}. So
far, much of the work has focused on
formalization~\cite{BenzakenCD14,CheneyU11,ShinnarSH15} or on
evaluating challenges involved in
mechanization~\cite{MalechaMSW10}. The closest related work is that of
Cherniack and Zdonik~\cite{cherniack1996rule,CherniackZ98}, which
focused on the formal specification of rule-based query optimizers and
used the Larch~\cite{Larch89} theorem prover to verify
correctness. Our work extends that approach in two ways: (i) we
describe an alternative combinator-based algebra with built-in support
for environments and (ii) we leverage recent advances in theorem
proving technology to specify a much larger part of the query
compiler.

How to best deal with variables and environments in algebraic
compilers has received relatively little attention. For SPJ
(Select-Project-Join) queries, variables can be eliminated at
translation time and equivalences can be simply defined for a given
static environment~\cite{aho1979efficient}. For query languages over
complex or nested data, reification of the environment as a record is
appealing in that existing relational techniques can be readily
applied. This idea has notably been used in algebraic compilers for
query languages over nested or graph data such as OQL~\cite{CluetM93}
and XQuery~\cite{MayHM04,re2006complete}. Full reification enables
relational optimizations, but can result in large or highly nested
plans in those languages as well. The algebra from~\cite{CluetM93}
does combine environments and reification, but assumes that
environments are fixed for the purpose of defining plan equivalence.

Dealing with bindings is also important for the formalization of
programming languages. The POPLmark
challenge~\cite{AydemirBFFPSVWWZ05} has helped spur an assortment of
techniques for representing and reasoning about bindings.  These are
all focused on traditional bindings, as introduced by functions.  Our
work uses explicit reified environments instead. It enables support
for the standard shadowing semantics for occurrences of a variable
while also supporting unification semantics, where the value of the
variable added to the environment has to be compatible with 
previous occurrences.

%%% Local Variables:
%%% TeX-master: "main"
%%% End:
