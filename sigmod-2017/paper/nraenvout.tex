\section{Translating from \NRAEnv}
\label{sec:export}

This section defines the translations of \NRAEnv to NRA and to the
Named Nested Relational Calculus (NNRC). The first translation is
useful to show that \NRAEnv shares the same expressiveness as NRA,
which is desirable since it establishes that we have not inadvertently
targeted a more expressive language. The second translation provides a
useful bridge to a representation with variables which can prove
useful e.g., for code generation. We will come back to that second
point in Section~\ref{section:implementation} where we describe our
implementation.

\paragraph*{From \NRAEnv\ to NRA}
We first consider the relationship between \NRAEnv and NRA.
%
Figure~\ref{fig:nraenvtonra} defines the translation function~$\etoa{q}$
from \NRAEnv to NRA.
%
It relies on the encoding of the two inputs of \NRAEnv~($\qID$ and
$\qENV$) as a record with two fields: \textit{D} for the input datum
and \textit{E} for the environment. This record is the single input
$\qID$ of NRA. Therefore, the translation of $\qID$ (resp. $\qENV$)
corresponds to accessing field \textit{D} (resp. \textit{E}).

This encoding surfaces in the translation of most \NRAEnv\ constructs.
%
For example, the translation of composition needs to reconstruct
the encoding of the input before the evaluation of the second part of
the query:
$$
\etoa{q_2 \circ q_1} = \etoa{q_2} \circ ([E:\qID.E] \opconcat [D:\etoa{q_1}])
$$
This translation lifts each element of the collection which is mapped
into a record containing the environment as field~$E$ and the element
of the collection as field~$D$.

The translation uses the unnest operator $\quntwo{A}{B}{q}$ defined
in Section~\ref{sec:nraenv:nra}. Unsurprisingly, this re-introduces some
of the nesting/complexity that was eliminated by supporting
environments in \NRAEnv. The correctness of the translation is
established by Theorem~\ref{thm:nraenvtonra-correctness}.

\begin{theorem}[\NRAEnv\ to NRA Correctness\,\coqdef{NRAEnv.Core.cNRAEnv}{unfold_env_nra}]
  \label{thm:nraenvtonra-correctness}
{  \small
  \begin{align*}
    \gamma \vdash q \rapp d_1\Downarrowa d_2 & \iff \vdash \etoa{q} \qapp \left([E:\gamma] \opconcat [D:d_1]\right)\Downarrown d_2
  \end{align*}}
\end{theorem}

We mentioned in Section~\ref{sec:nraenv:syntax} that NRA queries
have the same behavior evaluated with either NRA or \NRAEnv
semantics\,\coqdef{NRAEnv.Core.cNRAEnvIgnore}{cnraenv_of_nra}.
Therefore, in conjunction with
Theorem~\ref{thm:nraenvtonra-correctness}, we have a proof that
\NRAEnv\ has the same expressiveness as NRA.


\paragraph*{From \NRAEnv\ to NNRC}

We present the translation of \NRAEnv to the Named Nested
Relational Calculus (NNRC)~\cite{BusscheV07}, with a bag semantics.
The syntax of the calculus is\,\coqdef{NNRC.Core.cNNRC}{nnrc}:
\begin{gram}
e & ::= & x \mid d \mid \opunop e_1 \mid e_1 \opbinop e_2
\mid \elet{x}{e_1}{e_2}\\ &\mid& \efor{x}{e_1}{e_2} \mid \eif{e_1}{e_2}{e_3}
\end{gram}
Expressions can be variables~($x$), constants~($d$),
operators ($\opunop e_1$ or $e_1 \opbinop e_2$), dependent
sequencing~($\elet{x}{e_1}{e_2}$), bag
comprehensions~($\efor{x}{e_1}{e_2}$), or conditionals~($\eif{e_1}{e_2}{e_3}$).
%
The bag comprehension $\efor{x}{e_1}{e_2}$ constructs a bag where each
element is the result of the evaluation of $e_2$ in an environment in
which $x$ is bound to an element of the bag created by the evaluation
of~$e_1$.
%
We use the formal semantics of NNRC given
in~\cite{ShinnarSH15}\,\coqdef{NNRC.Core.cNNRC}{nnrc_core_eval}.

\begin{figure*}[th]
  \centering
  \begin{align*}
    \nton{x_{d},x_{e}}{d} &= d\\
    \nton{x_{d},x_{e}}{\qID} &= x_{d}\\
    \nton{x_{d},x_{e}}{\opunop q} &= \opunop \nton{x_{d},x_{e}}{q}\\
    \nton{x_{d},x_{e}}{q_1\opbinop q_2} &= \nton{x_{d},x_{e}}{q_1}\opbinop\nton{x_{d},x_{e}}{q_2}\\
    \nton{x_{d},x_{e}}{q_2\circ q_1} &= \elet{x}{\nton{x_{d},x_{e}}{q_1}}{\nton{x,x_{e}}{q_2}}&x\textrm{ is }\nfresh\\
    \nton{x_{d},x_{e}}{\qmap{q_2}{q_1}} &= \efor{x}{\nton{x_{d},x_{e}}{q_1}}{\nton{x,x_{e}}{q_2}}&x\textrm{ is }\nfresh\\
    \nton{x_{d},x_{e}}{\qselect{q_2}{q_1}} &=\opflatten\left(\efor{x}{\nton{x_{d},x_{e}}{q_1}}{\eif{\nton{x,x_{e}}{q_2}}{\{x\}}{\emptyset}}\right) &x\textrm{ is }\nfresh\\
    \nton{x_{d},x_{e}}{q_1\times q_2} &=
    \opflatten\left(\efor{x_1}{\nton{x_{d},x_{e}}{q_1}}{\efor{x_2}{\nton{x_{d},x_{e}}{q_2}}{x_1 \opconcat x_2}}\right)&x_1\textrm{ is }\nfresh\land x_2\textrm{ is }\nfresh\\
    \nton{x_{d},x_{e}}{\qdjoin{q_2}{q_1}} &=
    \opflatten\left(\efor{x_1}{\nton{x_{d},x_{e}}{q_1}}{\efor{x_2}{\nton{x_1,x_{e}}{q_2}}{x_1 \opconcat x_2}}\right)&x_1\textrm{ is }\nfresh\land x_2\textrm{ is }\nfresh\\
    \nton{x_{d},x_{e}}{q_1~\qor~q_2} &= \elet{x}{\nton{x_{d},x_{e}}{q_1}}{\left(\eif{\left(x=\emptyset\right)}{\nton{x_{d},x_{e}}{q_2}}{x}\right)}&x\textrm{ is }\nfresh\\
    \nton{x_{d},x_{e}}{\qENV} &= x_{e}\\
    \nton{x_{d},x_{e}}{q_2\circ^e q_1} &= \elet{x}{\nton{x_{d},x_{e}}{q_1}}{\nton{x_{d},x}{q_2}}&x\textrm{ is }\nfresh\\
    \nton{x_{d},x_{e}}{\qmapenv{q_2}} &= \efor{x}{x_{e}}{\nton{x_{d},x}{q_2}}&x\textrm{ is }\nfresh
  \end{align*}
  \caption{From \NRAEnv to NNRC\,\coqdef{Translation.cNRAEnvtocNNRC}{cnraenv_to_nnrc}. \qquad \fbox{\(\nton{x_{d},x_{e}}{q} = e\)}}
  \label{fig:tonrc-trans}
\end{figure*}

Figure~\ref{fig:tonrc-trans} defines the translation function
$\nton{x_{d},x_{e}}{q}$ from \NRAEnv to
NNRC.
%
The translation function is parameterized by two variables $x_{d}$ and
$x_{e}$ that are used to encode the input value and the environment.
%
So, for example, the translation of $\qID$ (resp. $\qENV$) returns the
corresponding variable $x_{d}$ (resp. $x_{e}$). This translation makes
explicit the handling of the input and the environment. For example,
in the translation of the composition the result of the evaluation of
the first expression becomes the input of the second expression:
$$
\nton{x_{d},x_{e}}{q_2\circ q_1} = \elet{x}{\nton{x_{d},x_{e}}{q_1}}{\nton{x,x_{e}}{q_2}} \qquad x\textrm{ is }\nfresh
$$

The translation of \NRAEnv to NNRC is similar to the translation of
NRA to NNRC presented
in~\cite{ShinnarSH15}\,\coqdef{Translation.NRAtocNNRC}{nra_to_nnrc}. However,
the two inputs of \NRAEnv can be translated directly to NNRC without encoding.
%
Both translations are proved
correct\,\coqdef{Translation.NRAtocNNRC}{nra_sem_correct}\coqdef{Translation.cNRAEnvtocNNRC}{nraenv_sem_correct}.


%%% Local Variables:
%%% TeX-master: "main"
%%% End:
